\documentclass{article}

\usepackage{fancyhdr}
\usepackage{extramarks}
\usepackage{amsmath}
\usepackage{amsthm}
\usepackage{amsfonts}
\usepackage{tikz}
\usepackage[plain]{algorithm}
\usepackage{algpseudocode}
\usepackage{hyperref}
\usepackage{listings}

\usetikzlibrary{automata,positioning}

%
% Basic Document Settings
%

\topmargin=-0.45in
\evensidemargin=0in
\oddsidemargin=0in
\textwidth=6.5in
\textheight=9.0in
\headsep=0.25in

\linespread{1.1}

\pagestyle{fancy}
\lhead{\hmwkAuthorName}
\chead{\hmwkClass \hspace{1cm} \hmwkTitle}
\rhead{\firstxmark}
\lfoot{\lastxmark}
\cfoot{\thepage}

\renewcommand\headrulewidth{0.4pt}
\renewcommand\footrulewidth{0.4pt}

\setlength\parindent{0pt}

%
% Create Problem Sections
%

\newcommand{\enterProblemHeader}[1]{
    \nobreak\extramarks{}{Problem \arabic{#1} continued on next page\ldots}\nobreak{}
    \nobreak\extramarks{Problem \arabic{#1} (continued)}{Problem \arabic{#1} continued on next page\ldots}\nobreak{}
}

\newcommand{\exitProblemHeader}[1]{
    \nobreak\extramarks{Problem \arabic{#1} (continued)}{Problem \arabic{#1} continued on next page\ldots}\nobreak{}
    \stepcounter{#1}
    \nobreak\extramarks{Problem \arabic{#1}}{}\nobreak{}
}

\setcounter{secnumdepth}{0}
\newcounter{partCounter}
\newcounter{homeworkProblemCounter}
\setcounter{homeworkProblemCounter}{1}
\nobreak\extramarks{Problem \arabic{homeworkProblemCounter}}{}\nobreak{}

%
% Homework Problem Environment
%
% This environment takes an optional argument. When given, it will adjust the
% problem counter. This is useful for when the problems given for your
% assignment aren't sequential. See the last 3 problems of this template for an
% example.
%
\newenvironment{homeworkProblem}[1][-1]{
    \ifnum#1>0
        \setcounter{homeworkProblemCounter}{#1}
    \fi
    \section{Problem \arabic{homeworkProblemCounter}}
    \setcounter{partCounter}{1}
    \enterProblemHeader{homeworkProblemCounter}
}{
    \exitProblemHeader{homeworkProblemCounter}
}

%
% Homework Details
%   - Title
%   - Due date
%   - Class
%   - Section/Time
%   - Instructor
%   - Author
%

\newcommand{\hmwkTitle}{Homework 1}
\newcommand{\hmwkClass}{ECE-GY 6913}
\newcommand{\hmwkStudentName}{Chang-Heng Liou}
\newcommand{\hmwkStudentEmail}{cl5533@nyu.edu}

%
% Title Page
%

\title{
    \vspace{2in}
    \textmd{\textbf{\hmwkClass\\ \hmwkTitle}}\\
    \vspace{0.1in}\large{\textit{\hmwkStudentName}}\\
    \vspace{0.1in}\large{\textit{\hmwkStudentEmail}}
    \vspace{3in}
}

\date{}

\renewcommand{\part}[1]{\textbf{\large Part \Alph{partCounter}}\stepcounter{partCounter}\\}

%
% Various Helper Commands
%

% Useful for algorithms
\newcommand{\alg}[1]{\textsc{\bfseries \footnotesize #1}}

% For derivatives
\newcommand{\deriv}[1]{\frac{\mathrm{d}}{\mathrm{d}x} (#1)}

% For partial derivatives
\newcommand{\pderiv}[2]{\frac{\partial}{\partial #1} (#2)}

% Integral dx
\newcommand{\dx}{\mathrm{d}x}

% Alias for the Solution section header
\newcommand{\solution}{\textbf{\large Solution}}

% Probability commands: Expectation, Variance, Covariance, Bias
\newcommand{\E}{\mathrm{E}}
\newcommand{\Var}{\mathrm{Var}}
\newcommand{\Cov}{\mathrm{Cov}}
\newcommand{\Bias}{\mathrm{Bias}}
\newcommand{\norm}[1]{\left\lVert#1\right\rVert}

\begin{document}

\maketitle

\pagebreak

\begin{homeworkProblem}
    (Basic) Translate the following to MIPS instructions, minimize the number of instructions:
    \begin{lstlisting}
        f = g + h + i + j;
        f = g + (h + 5);
    \end{lstlisting}
    If f, g, h, i, and j equal 1, 2, 3, 4, 5, then what is the final value of f?\\
    \\
    \textbf{Solution}\\
    \begin{lstlisting}
    .text
        main:
        li $t1, 2 # g
        li $t2, 3 # h
        li $t3, 4 # i
        li $t4, 5 # j
        add $t5, $t1, $t2 # g + h
        add $t6, $t3, $t4 # i + j
        add $t0, $t5, $t6 # f = (g + h) + (i + j)
        addi $t5, $t2, 5 # h + 5
        add $t0, $t1, $t5 # f = g + (h + 5)  
    \end{lstlisting}

\end{homeworkProblem}

\pagebreak

\begin{homeworkProblem}
    (Accessing Memory) Translate the following to MIPS instructions, minimize thenumber of instructions:
    \begin{lstlisting}
        f = g + h + B[4];
        g = g - A[B[4]];
    \end{lstlisting}
    Assume f, g, h, i, and j are assigned to \$s0, \$s1, \$s2, \$s3, and \$s4. 
    Also assume thebase (starting) address of arrays A and B are in registers \$s6 and \$s7.\\
    \\
    \textbf{Solution}\\
    \begin{lstlisting}
    .data
        A: .space 20
        B: .space 20
    .text
        main:
        la $t0, A # load address of array A
        la $t1, B # load address of array B
        add $s0, $s1, $s2 # f = g + h
        add $s0, $s0, 16($t1) # f = (g + h) + B[4]
        sub $s1, $s1, # DODO: A[B[4]]
    \end{lstlisting}
\end{homeworkProblem}

\pagebreak

\begin{homeworkProblem}
    (Loops and branches) Translate the following to MIPS instructions, minimize the number of instructions.
    You must use branch instructions, think about how to implement loops using the control (branch/jump) instructions. 
    Two loops are givenin C, implement them separately
    \begin{lstlisting}
        for (int i = 0; i < 10; i++) {
            a += b;
        }
        while (a < 10) {
            D[a] = b + a;
            a += 1;
        }
    \end{lstlisting}
    \textbf{Solution}\\
\end{homeworkProblem}

\pagebreak

\begin{homeworkProblem}
    (Recursive function) Implement factorial (code below) in the MIPS ISA. Pay careful attention to the return address register.
    \begin{lstlisting}
        int fact(int n) {
            if (n < 1) return 1;
            return (n x fact(n - 1));
        } 
    \end{lstlisting}
    \textbf{Solution}\\
\end{homeworkProblem}

\end{document}
